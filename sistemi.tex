\section{Dinamica di sistemi di oggetti}

\subsection{Equazioni cardinali della dinamica dei sistemi}

\begin{equation}
    \boxed{\vec{R}^{(e)} = \frac{d\vec{Q}}{dt}}
\end{equation}
\begin{equation}
    \boxed{\frac{d\vec{L}_{TOT}}{dt} = \vec{M}_{TOT}{^{(e)}}}
\end{equation}

\subsection{Urti - fenomeni d'urto tra due oggetti}
$\vec{Q}$ è costante durante l'urto
\subsubsection{Tipologie di urti}
\begin{itemize}
    \item   \textbf{Urti elastici} \\
            $K_{TOT}$ si conserva:
            \begin{equation*}
                \frac{1}{2}m_1v_1^{\tiny{(-)}^2} + 
                \frac{1}{2}m_2v_2^{\tiny{(-)}^2} = 
                \frac{1}{2}m_1v_1^{\tiny{(+)}^2} + 
                \frac{1}{2}m_2v_2^{\tiny{(+)}^2} 
            \end{equation*}
            $\to$ risolvibile solo in una dimensione
    \item   \textbf{Urti anleastici}
    
            \begin{equation*}
                K_{TOT}^{\tiny{(-)}} \neq K_{TOT}^{\tiny{(+)}}
            \end{equation*}
            $\to$ energia viene dispersa
    \item   \textbf{Urti perfettamente anelastici} \\
            I due oggetti si fondono in uno solo
            \begin{equation*}
                V^{\tiny{(+)}} = \frac{m_1v_1^{\tiny{(-)}} + m_2v_2^{\tiny{(-)}}}{m_1 + m_2}
            \end{equation*}
\end{itemize}


\subsection{Centro di massa}
La posizione del centro di massa è data dalla media pesata delle posizioni di ogni punto rispetto alla massa:
\begin{equation}
    \vec{r}_{\tiny{CM}} = \frac{\sum_{i=1}^N m_i \vec{r_i}}{\sum_{i=1}^N m_i}
\end{equation}

chiamando $\sum_{i=1}^N m_i$ $M$:
\begin{equation}
    \begin{cases}
        \vec{Q}=M\frac{d\vec{r}_{CM}}{dt} \\
        \vec{V}_{\tiny{CM}}=\frac{d\vec{r}_{CM}}{dt}
    \end{cases}
    \label{q e v}
\end{equation}
\begin{equation}
    \begin{cases}
        \frac{d\vec{Q}}{dt}=M\frac{d\vec{V}_{CM}}{dt} \\
        \vec{a}_{\tiny{CM}}=\frac{d^2\vec{r}_{CM}}{dt^2}
    \end{cases} 
    \label{dq e a}
\end{equation}

Ogni corpo può essere quindi studiato nel suo Centro di Massa, infatti da \eqref{q e v} e \eqref{dq e a}:
\begin{equation}
    \boxed{\vec{Q} = M\vec{V}_{\tiny{CM}}}
\end{equation}
\begin{equation}
    \boxed{\vec{R}^{(e)} = M\vec{a}_{\tiny{CM}}}
\end{equation}

Per un sistema isolato:
\begin{itemize}
    \item $\vec{R}^{(e)} = 0$
    \item $\vec{v}_{\tiny{CM}}$ costante
    \item $\vec{a}_{\tiny{CM}}$ costante
\end{itemize}

\subsection{Teoremi di König}
\begin{equation}
    \boxed{\vec{L}_{TOT_{(O)}} = \vec{r}_{CM} \times M \vec{v}_{CM} + \vec{L}_{TOT_{CM}}}
    \label{konig1}
\end{equation}
\vspace{\baselineskip}
\begin{equation*}
    K_{TOT} = \frac{1}{2}M_{TOT} \vec{V}_{\tiny{CM}}{^2} + \sum_{i=1}^N \frac{1}{2} m_i v_i{'}^2 
\end{equation*}
\begin{equation}
    \boxed{K_{TOT} = \frac{1}{2}M_{TOT} \vec{V}_{\tiny{CM}}{^2} + K_{TOT}{'}}
\end{equation}


\subsection{Lavoro ed Energia}

\begin{equation*}
    \Delta K_{TOT} = \work^{(i)}_{{i} \to {f}} + \work^{(e)}_{{i} \to {f}}
\end{equation*}

$V \to$ energia potenziale interna al sistema 

\begin{equation*}
    \work^{(i)}_{{i} \to {f}} = \work^{(inc)}_{{i} \to f} - \Delta V^{(i)}
\end{equation*}

\begin{equation*}
    \work^{(e)}_{{i} \to {f}} = \work^{(enc)}_{{i} \to f} - \Delta V^{(e)}
\end{equation*}
\vspace{\baselineskip}
\begin{equation*}
    E_m = K_{TOT} + V^{(i)} + V^{(e)}
\end{equation*}

\begin{equation*}
    \Delta E_m = \work^{(inc)}_{{i} \to {f}} + \work^{(enc)}_{{i} \to {f}}
\end{equation*}


In assenza di forze non conservative
\begin{equation*}
    \Delta E_m = 0
\end{equation*}

\vspace{\baselineskip}

\begin{equation*}
    K_{TOT} = K_{CM} + K_{TOT}{'}
\end{equation*}

\begin{equation*}
    K_{CM} + K_{TOT}{'} + \Delta V^{(i)} = \work^{(e)}_{i \to f}
\end{equation*}

Introducendo l'energia interna del sistema $U=K_{TOT}{'} + \Delta V^{(i)}$
\begin{equation}
    \Delta K_{CM} + \Delta U = \work^{(e)}_{i \to f}
\end{equation}