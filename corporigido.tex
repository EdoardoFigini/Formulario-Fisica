\section{Sistemi Rigidi (Corpo Rigido)}
La distanza tra due oggetti puntiformni qualsiasi del sistema rimane costante nel tempo

\begin{equation*}
    \vec{v} = \vec{V_{(O')}} + \omega \times (\vec{r} - \vec{R_{(O')}})
\end{equation*}

\begin{equation}
    \vec{v_P} = \vec{V}_{CM} + \omega \times (\vec{r_P} - \vec{r}_ {CM})
\end{equation}

\begin{itemize}
    \item   Se $O'$ è in quiete rispetto a O
            \begin{equation*}
                \vec{V}_{(O')} = 0
            \end{equation*}
            \begin{equation}
                \frac{d\vec{L}_{TOT_{(O')}}}{dt} = \vec{M}_{(O')}^{(e)}
            \end{equation}
    \item   Se $O' \equiv CM$  
            \begin{equation*}
                \vec{V}_{(O')} = \vec{V}_{CM} \;\; \to \;\; \vec{V}_{(O')} \times M \vec{V}_{CM} = 0
            \end{equation*}
            \begin{equation*}
                \frac{d\vec{L}_{TOT_{CM}}}{dt} = \vec{M}_{CM}^{(e)}
            \end{equation*}
            
            \begin{equation*}
                \vec{L}_{TOT_{CM}} = \vec{L}_{TOT_{(O')}} - \vec{r}_{CM} \times M \vec{v}_{CM}
            \end{equation*}
            \begin{equation}
                \vec{L}_{TOT_{(O)}} = \vec{r}_{CM} \times M \vec{v}_{CM} + \vec{L}_{TOT_{CM}}
            \end{equation}
            I teorema di König \eqref{konig1}
\end{itemize}

\subsection{Momento di Inerzia}
Chiamando $r_i$ la distanza del punto $m_i$ dall'asse di rotazione e $\rho$ la densità:
\begin{equation}
    I = \sum_{i=1}^{N} m_i r_i^2
    \label{inerzia}
\end{equation}
\begin{equation}
    \rho(P) = \frac{dm}{dV}
    \label{densita}
\end{equation}

Da \eqref{inerzia} e \eqref{densita}:
\begin{equation*}
    I = \lim_{\Delta V \to 0} \sum_{i=1}^{N} \Delta m_i r_i^2
\end{equation*}
\begin{equation}
    \boxed{I = \iiint r^2 \rho dV} 
\end{equation}

\subsubsection{Teorema di Huygens-Steiner}
Per trovare $I$ su un asse parallelo:
\begin{equation}
    \boxed{I = I_{CM} + M d^2}
    \label{teoHS}
\end{equation}
dove $d$ è la distanza tra i due assi


\subsection{Sistemi rigidi in moto}
\subsubsection{Moto di pura traslazione}
\begin{equation*}
    \vec{Q} = M \vec{V}_{CM}
\end{equation*}
\begin{equation}
    \vec{L}_{TOT_{(O)}} = \vec{r}_{CM} \times \vec{Q}
\end{equation}
\subsubsection{Moto di pura rotazione}
$\vec{r_i}$ scomposto in componenti: 
\begin{itemize}
    \item $z_i \vec{u_z}$ lungo asse $z$
    \item $\rho_i$ perpendicolarmente a $z$
\end{itemize}
\begin{equation}
    \vec{L}_{TOT_{(O)}} = I \vec{\omega} - \sum_{i=1}^{N} m_i z_i \omega \vec{\rho_i}
\end{equation}
Se asse di rotazione è anche asse di simmetria:
\begin{equation}
    \boxed{\vec{L}_{TOT_{(O)}} = I \vec{\omega}}
\end{equation}

considerata $\vec{M_z}$ la proiezione di $\vec{M}_{(O)}^{(e)}$ lungo $z$:
\begin{equation}
        \boxed{\vec{M}_z^{(e)} = I \vec{\alpha}}
\end{equation}

\subsubsection{Moto di rotolamento}
Moto studiato nel punto di contatto del sistema con il suolo $Q$, chiamato \emph{centro di istantanea rotazione}

\begin{equation}
    \vec{v} = \vec{\omega} \times \vec{r}
\end{equation}

\begin{equation}
    |\vec{v}_{CM}| =\omega R
    \label{velocitarotolamento}
\end{equation}

da \eqref{velocitarotolamento}:

\begin{equation}
    |\vec{Q}| = M R \omega
\end{equation}

\subsection{Statica di corpi rigidi}
Condizioni necessarie per la quiete:

\begin{equation*}
    R^{(e)} = 0
\end{equation*}

\begin{equation*}
    M^{(e)}_{TOT_{(O')}}= 0
\end{equation*}

\subsection{Energia Cinetica e Lavoro delle Forze esterne}

\begin{equation}
    \boxed{K_{TOT} = \frac{1}{2} M \vec{v}_{CM}{^2} + \frac{1}{2} I \omega^2}
\end{equation}

\vspace{\baselineskip}

\begin{equation*}
    \work_{i \to f}^{(e)} = \underset{\gamma_{CM}}{\int_{A_{CM}}^{B_{CM}}}\vec{R}^{(e)} d\vec{r}_{CM} + \int_{\theta_{B}}^{\theta_{A}}\vec{M}_z^{(e)} d\theta = \Delta K_{TOT}
\end{equation*}

\begin{equation}
    \boxed{\work_{i \to f}^{(e)} = \Delta K_{CM} + \Delta K_{ROT} = \Delta K_{TOT}}
\end{equation}